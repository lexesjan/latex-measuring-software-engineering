\documentclass{article}

\usepackage{xurl}
\usepackage{hyperref}
\usepackage{graphicx}

\title{Measuring Software Engineering}
\author{Lexes Jan Mantiquilla}
\date{\today}

\begin{document}

\maketitle
\newpage

\tableofcontents
\newpage

\section{Introduction}
This report will focus on explaining the various ways in which one can measure
the software engineering process. The report outline what platforms or
frameworks are available to perform measurement. As well as that, the report
will go through the algorithmic approaches and the ethic concerns surrounding
the measurement of the software engineering process.

\section{How software engineering can be measured}

\section{Platforms available}
Measuring the software engineering process is a valuable process. Thus, many
services have been created to aid companies analyze and display this process.
The data provided by the services help employers visualize whether or not the
ongoing projects will be delivered on time, see which employees have a high
impact on the company, etc.

\subsection{Pluralsight Flow}
Pluralsight Flow, called GitPrime previously, is a service which obtains data
from git host, e.g. GitHub, and the git information from repositories which the
engineers work on.~\cite{plural2019sight} The said data is processed to produce
useful metrics to indicate the performance of software engineers.

Version Control Systems like git generate a large digital footprint for the
developers that use it. Information like the number of commits done by a
developer, the lines of code changed by a developer, is produced through the
use of a version control system. These metrics alone may not be a good
indication of the productivity of a software engineer. However, when processed
together, a clearer picture is obtained.

Pluralsight Flow, for example, will use the information found pull requests
(PRs), tickets linked to said PRs and commits to produce useful graphs and
metrics.

\subsubsection{New metrics extrapolated by Pluralsight Flow:}
\begin{itemize}
  \item Churn --- The amount of code which is changed or deleted shortly after
    it was written. A high amount of churn may indicate that something may be
    affecting the development life cycle.
  \item Efficiency --- A percentage evaluating whether the committed code is
    productive work. This metric is directly related to the churn metric.
  \item Impact --- A score given to a commit which measures the effect of
    changing or adding specific lines of code. This metric uses many factors to
    determine the score such as the amount of code changed, the number of files
    affected, the percentage of old code edited and many more.
\end{itemize}
~\cite{plural2019sight2}

\subsection{Azure DevOps services}
Azure DevOps services is a collection of developer services provided by
Microsoft to facilitate the software engineering process. There are many Azure
DevOps services available however this report focus on the Azure Boards and the
Azure Pipelines service in particular.~\cite{azure2020devops}

\subsubsection{Azure Boards}
Azure Boards is a service which allows software engineers to track and manage
their work for software projects. Azure Boards is built with software
engineering processes in mind and has native support for Kanban and Scrum. It
offers built-in scrum boards and and tools to help the team run sprints and
stand-ups.

The basis of Azure Boards is an entity called the `work item'. This entity
tracks all the work done for the software project. A `work item' communicates
the details of the work to be completed to the team. Each `work item' has a
status attached which is an indication the progress of the work. This allows
the whole team to be on the same page with regards to what work is assigned to
who and whether or not it has been completed.

Each `work item' has a rich set of information attached to it. It contains
information such as links to discussions regarding the `work item', any
commits, PRs or tickets related, as well as a history of all changes to the
`work item'.

The information collected through the use of Azure Boards is analyzed and
presented on a dashboard which indicates the health of the project. The
dashboard contains all the metrics gathered, such as the velocity of the
project or average completion time, in one place.~\cite{azure2020boards}

\subsubsection{Azure Pipelines}
Azure Pipelines is a cloud continuous integration (CI) and continuous delivery
(CD) service. That is, a service which builds, tests and deploys your code
automatically. This service is used in conjunction with a git host such as
GitHub or Azure Repos.

The metrics which Azure Pipelines provides are the pipeline pass rate, the code
test pass rate and the average pipeline duration. The information for the
pipeline analytics is obtained from the pipeline runs.

\section{Algorithmic approaches}

\section{Ethics}

\section{Conclusion}

\newpage

\bibliographystyle{plain}
\bibliography{bibliography.bib}

\end{document}
