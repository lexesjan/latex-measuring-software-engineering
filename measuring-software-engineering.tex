\documentclass{article}

\usepackage{xurl}
\usepackage{hyperref}

\title{Measuring Software Engineering}
\author{Lexes Jan Mantiquilla}
\date{\today}

\begin{document}

\maketitle
\tableofcontents
\newpage

\section{Introduction}
In this report I will explain the various ways in which one can measure the
software engineering process. I will outline what platforms or frameworks are
available to perform measurement. As well as, that I will go through the
algorithmic approaches and the ethic concerns surrounding the measurement of
the software engineering process.

\section{How software engineering can be measured}

\section{Platforms available}
Measuring the software engineering process is a valuable process. Thus, many
services have been created to aid companies analyse and display this process.
The data provided by the services help employers visualize whether or not the
ongoing projects will be delivered on time, see which employees have a high
impact on the company, etc.

\subsection{Pluralsight Flow}
Pluralsight Flow, called GitPrime previously, is a service which obtains data
from git host, e.g. GitHub, and the git information from repositories which the
engineers work on.~\cite{plural2019sight} The said data is processed to produce
useful metrics to indicate the performance of software engineers.

Version Control Systems like git generate a large digital footprint for the
developers that use it. Information like the number of commits done by a
developer, the lines of code changed by a developer, is produced through the
use of a version control system. These metrics alone may not be a good
indication of the productivity of a software engineer. However, when processed
together, a clearer picture is obtained.

Pluralsight Flow, for example, will use the information found pull requests
(PRs), tickets linked to said PRs and commits to produce useful graphs and
metrics.

\subsubsection*{Some examples new metrics created by Pluralsight Flow are:}
\begin{itemize}
  \item Churn --- The amount of code which is changed or deleted shortly after
    it was written. A high amount of churn may indicate that something may be
    affecting the development lifecycle.
  \item Efficiency --- A percentage evaluating whether the committed code is
    productive work. This metric is directly related to the churn metric.
  \item Impact --- A score given to a commit which measures the effect of
    changing or adding specific lines of code. This metric uses many factors to
    determine the score such as the amount of code changed, the number of files
    affected, the percentage of old code edited and many more.
\end{itemize}
~\cite{plural2019sight2}

\section{Algorithmic approaches}

\section{Ethics}

\section{Conclusion}

\newpage

\bibliographystyle{plain}
\bibliography{bibliography.bib}

\end{document}
